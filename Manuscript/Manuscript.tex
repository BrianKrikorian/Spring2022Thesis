\documentclass[12pt]{article}
\usepackage[margin=1in]{geometry}

\usepackage{xr-hyper}
\RequirePackage[colorlinks,citecolor=blue,urlcolor=blue]{hyperref}
\usepackage{listings}
\usepackage{rotating, graphicx}
\usepackage{booktabs, natbib}

\usepackage{amsmath, enumerate, quoting, subfigure, amssymb, url}

\usepackage{color}
\newcommand{\red}[1]{\textcolor{red}{#1}}
\newcommand{\blue}[1]{\textcolor{blue}{#1}}
\newcommand{\gv}[1]{\textcolor{blue}{(GV: #1)}}
\DeclareMathOperator*{\argmin}{arg\,min}
\DeclareMathOperator*{\cov}{Cov}
\DeclareMathOperator*{\corr}{Corr}

\usepackage{xr}
%\externaldocument{isgee}
%\externaldocument{supp}

% remove left indentation in itemize
\usepackage{enumitem}
\setlist[itemize]{leftmargin=*}

\usepackage{xcolor}
\newcommand{\jy}[1]{\textcolor{red}{(JY: #1)}}

% graph path
\graphicspath{{./figs/}}

\usepackage{geometry}
\geometry{vmargin={1in,1in}, hmargin={.75in, .75in}}

\usepackage[labelfont={footnotesize,bf} , textfont=footnotesize]{caption}
\captionsetup{labelformat=simple, labelsep=period}
\newcommand\num{\addtocounter{equation}{1}\tag{\theequation}}
\renewcommand{\theequation}{\arabic{equation}}
\makeatletter
\renewcommand\tagform@[1]{\maketag@@@ {\ignorespaces {\footnotesize{\textbf{Equation}}} #1.\unskip \@@italiccorr }}
\makeatother
\setlength{\intextsep}{10pt}
\setlength{\abovecaptionskip}{2pt}
\setlength{\belowcaptionskip}{-10pt}
\renewcommand{\textfraction}{0.10}
\renewcommand{\topfraction}{0.85}
\renewcommand{\bottomfraction}{0.85}
\renewcommand{\floatpagefraction}{0.90}

\quotingsetup{leftmargin = 0pt}

\newenvironment{comment}%
{\begin{quoting}\noindent\small\it\ignorespaces%
  }{\end{quoting}}
 \title{Points Above Replacement: A New NBA Metric to Evaluate Player Performance}
\author{
Brian Krikorian*, Jun Yan \\ \medskip 
Department of Statistics, University of Connecticut, Storrs, CT  \\  \medskip 
Students: brian.krikorian@uconn.edu* \\
Mentor: jun.yan@uconn.edu 
}
\begin{document}

% Abstracts are required.
\section*{Abstract}
In attempting to predict the score of an NBA game, players who are injured, suspended, or for any other 
reason not playing have a significant impact on said prediction. Baseball uses a statistic called Wins Above 
Replacement, or WAR, to tell how many wins a team would gain or lose if they replaced a certain player 
with one who is league average. The goal of this paper is to create a similar statistic, called Points Above 
Replacement, or PAR, that uses statistics from throughout the season to predict how many points a team 
gains or loses by missing a certain player, and thus having to replace him with one who is "league 
average". The statistic is standardized and adjusted by both position and value to a player's given team. 
The model roughly follows the calculation of WAR in baseball, assigning values to points created on offense 
and defense, as well as a positional and team adjustment. A comparison with the created metric along with 
other well-known advanced player metrics is done via a multiple regression.

% Keywords are required.
\section*{Keywords} 
NBA; Basketball; Player Performance; Basketball Metrics;
\vspace{.12 in}
% Start the main part of the manuscript here.
% Comment out section headings if inappropriate to your discipline.
% If you add additional section or subsection headings, use an asterisk * to avoid numbering. 

\section{Introduction}
%Why is this important/interesting? What has been done before on this topic? (Literature Review) What am I introducing that is new? 
There exists many prediction models for NBA basketball, many with the purpose of anticipating the final 
score of a given game. However, the majority of these models use team average statistics to base their 
decision, similar to the openWAR method in baseball introduced by Greg Matthews and others. However, 
teams rosters can change every night, and do often. When star players are ruled out, players of lower 
quality take their minutes, and are expected to perform at a lower level. When lower level players do not 
play in a given game, star players are expected to play more minutes, which most likely will cause a slight 
uptick in performance. This reasoning is the cause for the creation of the new metric, PAR. The measure of 
how many points a team will gain or lose if a certain player does not play. The metric is created based on 
the last completed season, the 2020-2021 season, which was cut to 72 games, as opposed to 82, due to 
the COVID-19 pandemic. Future models could be adjusted to a full 82 game season, and automated to 
update statistics as the season goes on. Position and team adjustments allow for a player's value to his 
specific team to be measured, giving a better expectation for their presence or absence. Adjusting to a 
position allows one to analyze a player based on what his position requires. A point guard is expected to 
pass, shoot three pointers, and get steals more than he is expected to block shots and rebound. The 
opposite is true for centers, so different coefficients are created based on the primary position a player 
played throughout the season. The team adjustment is a function of how many minutes a player played 
throughout the entire season, and how many of their team's total points were scored by them. If a player 
plays a majority of his team's minutes, and scores a relatively high percentage of his team's points, he will 
be more valuable to his team, and his team will be hurt more by his absence in a game.



\section{Data}
% Include basic graphics here?
All basic and advanced statistics from the 2020-2021 season were collected from the Basketball Reference 
website. Each teams total points throughout the course of the season were also collected from the 
Basketball Reference website. From the basic statistics, 4 metrics were created in order to calculate PAR. 
Offensive Points is a function of how many points a player will create on offense. While the coefficients vary 
by position, the statistics used are effective field goal percentage, three point percentage, free throw 
percentage, offensive rebounds, assists, points, and turnovers. Similarly, Defensive Points is a function of 
how many points a player helps prevent while his team is on defense. The coefficients also vary by position, 
but this metric is based on defensive rebounds, total rebounds, steals, blocks, and personal fouls. The 
Positional Adjustment metric creates a coefficient of how valuable a player is to their individual team, and is 
based on how many minutes a player plays relative to their team, as well as how many points they score 
relative to their teams total points. The last statistic created is Points Per Win, which is based on the 
average amount of points a player will create in a game. This function is based on the sum of Offensive and 
Defensive Points, and is scaled to incorporate the total number of games, minutes, and players in a season. 
This statistic is very similarly calculated to Runs Per Win in the WAR statistic. Given that over the course of 
a 72 game season, all teams will play each other at least once, we assume a relatively equal distribution in 
terms of strength of schedule. Advanced statistics for each player were also collected in order to perform a 
regression, to see how PAR compares with known and accepted metrics. Value Over Replacement Player, 
also known as VORP, was one of the metrics used, and is calculated with the help of Box Score Plus 
Minus, also known as BPM. These statistics also help give a relative estimate of a player's value compared 
to that of an average player, but they translate the statistic to a league average team as well. PAR does not 
do this, because it is concerned with how valuable a player is to his specific team, not throughout the entire 
league. If a player switched teams, his PAR value would most likely change, as his importance to one team 
may be more or less than it would to another. For simplicity, the team that a player played his most games 
on was taken into consideration for this metric. Other statistics used include Player Efficiency Rating, a 
per-minute 
rating of player performance, and Win Shares, which estimates the number of wins a player creates 
for his team throughout the season. Due to all of these metrics' relative similarity to PAR, it is suggested 
that if they are highly correlated with PAR, it is a successful metric. However, we expect some differentiation 
between the metrics, as they all vary slightly in what they attempt to convey. The relative correlation 
between all these statistics shows that PAR is related to other advanced metrics analyzing player efficiency, 
but also that it introduces something new in basketball statistics.



\section{Methods}
% Should dive more in depth about how everything was obtained, calculated, etc.
The basic method for calculating the PAR method was based on how WAR is calculated in baseball, but 
also took influences from other advanced basketball statistics. WAR is based on Offensive and Defensive 
Runs, which are calculated subjectively via attributes of the game such as hitting, base running, fielding, 
and so forth. The subjectivity is in the coefficients chosen to scale statistics which impact Offensive and 
Defensive Runs. Thus, there are actually an infinite different kinds of WAR in baseball, depending on which 
method of calculating Offensive and Defensive Runs is used. Similarly, PAR is created by calculating 
Offensive and Defensive Points statistics, to analyze how many points a player adds for his team, and how 
many he prevents the other team from scoring. Using the basic offensive statistics of Effective Field Goal 
Percentage, Three Point Percentage, Free Throw Percentage, Offensive Rebounds, Assists, Points, and 
Turnovers, we assign a coefficient to them based on how important they are for a given position. Point 
guards are expected to pass more than centers, and a point guard's passing ability has more value to his 
team than a center, so more weight is placed on that statistic for point guards. This goes for all positions, 
and a coefficient matrix of a constant scale is created. That is to say, assists have a coefficient from point 
guards, shooting guards, small forwards, power forwards, and centers, respectively, of 4, 3.5, 3, 2.5, and 2. 
The coefficient by position method is one similar to that of Box Plus Minus, another advanced NBA metric. 
The position a player was assigned for the calculation was the position where he played the most minutes 
throughout the 2020-2021 season. This coefficient method gives a clean scaling into the calculation of 
Offensive Points. Some statistics, such as Points and Effective Field Goal Percentage, are equally 
important, regardless of the position of a given player. Thus, these statistics have a constant coefficient that 
does not change. Once all coefficients were applied to all statistics, the sum was taken to give Offensive 
Points. The exact same method was applied to the Defensive Points calculation, but used the basic 
defensive statistics of Defensive Rebounds, Total Rebounds, Steals, Blocks, and Personal Fouls. As 
turnovers and personal fouls are not desirable amongst players, the coefficients for these statistics is 
negative, which allows for a scale of not only the good a player does, but also what they do poorly. This 
coefficient method is similar to the Positional Adjustment metric applied in WAR calculation, which seeks to 
adjust a player's importance based on what is expected of their position.

The next step was creating a Team Adjustment statistic. As stated before, the PAR metric seeks to 
anticipate how valuable a player is to their given team, rather than as an average of the entire league. This 
gives more of an insight of how a team will fare without a certain player. We expect this statistic to change if 
a player were to switch teams. This is similar to the League Adjustment in WAR calculation, where a 
player's value is adjusted based on whether he plays in the American League or the National League. The 
Eastern and Western Conferences in the NBA have much less disparity compared to the MLB, and each 
conference plays each other much more frequently, so we instead change this to adjust a player's value to 
his specific team. The team assigned to a given player was the team where he played the most games 
during the 2020-2021 season. If a player had played an even amount of games for 2 separate teams, the 
team the player had played the most minutes for was assigned as his team. The Team Adjustment statistic 
is then calculated by taking the total minutes a player was on the court during the entire season, and diving 
it by a constant of 3,456 (the maximum number of minutes a player could have played during the shortened 
season, 48 minutes per game multiplied by 72 games). Then, the percentage of points a player scored for 
his team (player's total points divided by team's total points throughout the season) was calculated, and the 
two results were multiplied together. As a player cannot play 100 percent of his team's minutes, or score 
100 percent of his team's points, the result is a number between 0 and 1. A similar adjustment is also used 
in the calculation of Box Plus Minus, which is a key factor in calculating Value Over Replacement Player, 
two of the advanced metrics PAR will be compared against. The goal of this adjustment is to give less of a 
penalty to players who play a majority of their team's minutes, and score a majority of their team's points. 
This helps truly elite players stand out amongst the pack, as well as players who may not be the best of the 
best, but matter more to their team specifically than in the grand scheme of things.

Lastly, WAR implements a Runs Per Win statistic, which gives a basic idea of how many runs are needed 
for a win in the MLB. Similarly, a Points Per Win statistic is created for the NBA, which allows a scaling 
down of the PAR statistic. We calculate this via the combined Offensive and Defensive Points statistic for a 
player divided by the number of minutes he played, then adding 2,160 (72 games per season times 30 
teams in the NBA, thus, the total number of chances to win throughout the season). This gives a basic 
metric of a players points created as a function of how many minutes he played, and allows us to even the 
metric out a bit more from star players to role players. This statistic serves as the denominator of our PAR 
calculation.

Once all supporting statistics of PAR have been created, all that is left to do is combine them to achieve our PAR values for each player in the NBA. The formula for PAR is as follows:

\begin{equation}
((Offensive Points + DefensivePoints) * (Team Adjustment))/(Points Per Win)
\end{equation}

This gives a good idea of how many points a player accounts for during a game, how important that is for 
his given team, and how important it is compared to all other players. The last step is to standardize this 
PAR statistic, so that the mean is 0, and the standard deviation is 1. Logically, this helps the idea of the 
metric in the sense that a perfectly league average player should have a PAR of 0, and his presence or 
absence from his team neither adds nor subtracts value. Thus, we have created a new NBA player metric, 
Points Above Replacement, PAR.
 
 
\section{Results}
% Multiple Regression
% Analyze Graphs of PAR, filter by position perhaps?
% Any other tests that should be done?
First, the top 10 players in terms of PAR in the league will be analyzed. Then, the top 10 PAR values at each position will be evaluated.

\begin{table}[tbp]
  \caption{PAR Values from the Entire NBA}
  \label{tab:NBAtable}
\centering
\begin{tabular}[t]{lccllll}
  \toprule
  Rank & Player & Position & Team & PAR\\
  \midrule
 1 & Nikola Jokic & C & Denver Nuggets & 7.69\\
  \midrule
 2 & Nikola Vucevic & C & Orlando Magic & 6.55\\
  \midrule
 3 & Julius Randle & PF & New York Knicks & 6.49\\
  \midrule
 4 & Luka Doncic & PG & Dallas Mavericks & 4.61\\
  \midrule
 5 & Damian Lillard & PG & Portland Trail Blazers & 4.47\\
  \midrule
 6 & Giannis Antetokounmpo & PF & Milwaukee Bucks & 4.21\\
  \midrule
 7 & Stephen Curry & PG & Golden State Warriors & 4.17\\
  \midrule
 8 & Russell Westbrook & PG & Washington Wizards & 4.05\\
  \midrule
 9 & Jayson Tatum & SF & Boston Celtics & 3.96\\
  \midrule
 10 & Domanatas Sabonis & PF & Indiana Pacers & 3.21\\
  \bottomrule
\end{tabular}
\end{table}

As can be seen, one player rises above all others in both PAR and VORP. That player is Nikola Jokic of the 
Denver Nuggets, and this "outlier" makes perfect sense, and helps show just how dominant his MVP 
season was. Not only did Jokic average 26.4 points, 10.8 rebounds, and 8.3 assists per game, he was 
doing it from the center position. Having a center that has the ability to perform in all aspects of the game, 
even those that his position does not require, proves his value to his team, which is why his value is so 
much higher in both the new PAR metric, and the accepted VORP metric. This logic applies to Nikola 
Vucevic, Julius Randle, Russell Westbrook, and Domanatas Sabonis as well. While they might not be 
considered the best at their position, or an extremely elite player in the NBA, they are very versatile players, 
that can do anything on the court, and thus they are rewarded with a high PAR value. This makes sense, as 
without these players, a lot of the statistics their team accumulates are gone, so while they may not have as 
much value in the grand scheme of the NBA, they were very much the most valuable players for their 
teams. Another key thing to note is that some players, such as LeBron James, have a lower PAR than 
expected, and this could be due to the fact that they played most frequently in a position that did not match 
their skill set. LeBron James played primarily as a point guard, but his skillset matches that of a small 
forward or power forward. Thus, his strengths are not adequately measured, as he is essentially being 
played "out of position".

\begin{table}[tbp]
  \caption{PAR Values from all Point Guards}
  \label{tab:PGtable}
\centering
\begin{tabular}[t]{lccllll}
  \toprule
  Rank & Player & Team & PAR\\
  \midrule
 1 & Luka Doncic & Dallas Mavericks & 4.61\\
  \midrule
 2 & Damian Lillard & Portland Trail Blazers & 4.47\\
  \midrule
 3 & Stephen Curry & Golden State Warriors & 4.17\\
  \midrule
 4 & Russell Westbrook & Washington Wizards & 4.05\\
  \midrule
 5 & Trae Young & Atlanta Hawks & 2.69\\
  \midrule
 6 & De'Aaron Fox & Sacramento Kings & 1.94\\
  \midrule
 7 & Chris Paul & Phoenix Suns & 1.93\\
  \midrule
 8 & Kyrie Irving & Brooklyn Nets & 1.74\\
  \midrule
 9 & Dejounte Murray & San Antonio Spurs & 1.64\\
  \midrule
 10 & Ja Morant & Memphis Grizzlies & 1.45\\
  \bottomrule
\end{tabular}
\end{table}

\begin{table}[tbp]
  \caption{PAR Values from all Shooting Guards}
  \label{tab:SGtable}
\centering
\begin{tabular}[t]{lccllll}
  \toprule
  Rank & Player & Team & PAR\\
  \midrule
 1 & Bradley Beal & Washington Wizards & 3.17\\
  \midrule
 2 & Devin Booker & Phoenix Suns & 2.74\\
  \midrule
 3 & Terry Rozier & Charlotte Hornets & 2.59\\
  \midrule
 4 & Zach Lavine & Chicago Bulls & 2.29\\
  \midrule
 5 & RJ Barrett & New York Knicks & 2.28\\
  \midrule
 6 & Anthony Edwards & Minnesota Timberwolves & 2.22\\
  \midrule
 7 & Jaylen Brown & Boston Celtics & 1.87\\
  \midrule
 8 & Collin Sexton & Cleveland Cavaliers & 1.85\\
  \midrule
 9 & Buddy Hield & Sacramento Kings & 1.72\\
  \midrule
 10 & Norman Powell & Toronto Raptors & 1.37\\
  \bottomrule
\end{tabular}
\end{table}

\begin{table}[tbp]
  \caption{PAR Values from all Small Forwards}
  \label{tab:SFtable}
\centering
\begin{tabular}[t]{lccllll}
  \toprule
  Rank & Player & Team & PAR\\
  \midrule
 1 & Jayson Tatum & Boston Celtics & 3.96\\
  \midrule
 2 & Khris Middleton & Milwaukee Bucks & 2.39\\
  \midrule
 3 & Brandon Ingram & New Orleans Pelicans & 2.13\\
  \midrule
 4 & Kawhi Leonard & Los Angeles Clippers & 1.58\\
  \midrule
 5 & Paul George & Los Angeles Clippers & 1.44\\
  \midrule
 6 & Michael Porter Jr. & Denver Nuggets & 1.42\\
  \midrule
 7 & Jimmy Butler & Miami Heat & 1.41\\
  \midrule
 8 & Bojan Bogdonavic & Utah Jazz & 1.35\\
  \midrule
 9 & Mikal Bridges & Phoenix Suns & 1.19\\
  \midrule
 10 & Jerami Grant & Detroit Pistons & 1.14\\
  \bottomrule
\end{tabular}
\end{table}

\begin{table}[tbp]
  \caption{PAR Values from all Power Forwards}
  \label{tab:PFtable}
\centering
\begin{tabular}[t]{lccllll}
  \toprule
  Rank & Player & Team & PAR\\
  \midrule
 1 & Julius Randle & New York Knicks & 6.49\\
  \midrule
 2 & Giannis Antetokounmpo & Milwaukee Bucks & 4.21\\
  \midrule
 3 & Domanatas Sabonis & Indiana Pacers & 3.21\\
  \midrule
 4 & Zion Williamson & New Orleans Pelicans & 3.07\\
  \midrule
 5 & Andrew Wiggins & Golden State Warriors & 2.26\\
  \midrule
 6 & Demar Derozan & San Antonio Spurs & 1.88\\
  \midrule
 7 & Tobias Harris & Philadelphia 76ers & 1.87\\
  \midrule
 8 & Pascal Siakam & Toronto Raptors & 1.72\\
  \midrule
 9 & John Collins & Atlanta Hawks & 1.35\\
  \midrule
 10 & Harrison Barnes & Sacramento Kings & 1.11\\
  \bottomrule
\end{tabular}
\end{table}

\begin{table}[tbp]
  \caption{PAR Values from all Centers}
  \label{tab:Ctable}
\centering
\begin{tabular}[t]{lccllll}
  \toprule
  Rank & Player & Team & PAR\\
  \midrule
 1 & Nikola Jokic & Denver Nuggets & 7.69\\
  \midrule
 2 & Nikola Vucevic & Orlando Magic & 6.55\\
  \midrule
 3 & Rudy Gobert & Utah Jazz & 3.10\\
  \midrule
 4 & Bam Adebayo & Miami Heat & 2.78\\
  \midrule
 5 & Joel Embiid & Philadelphia 76ers & 2.26\\
  \midrule
 6 & Clint Capela & Atlanta Hawks & 2.25\\
  \midrule
 7 & DeAndre Ayton & Phoenix Suns & 2.11\\
  \midrule
 8 & Jonas Valanciunas & Memphis Grizzlies & 1.98\\
  \midrule
 9 & Karl-Anthony Towns & Minnesota Timberwolves & 1.82\\
  \midrule
 10 & Kelly Olynyk & Miami Heat & 1.39\\
  \bottomrule
\end{tabular}
\end{table}

PAR will also be compared with the 4 accepted advanced metrics that were used as regressors in the 
multiple regression on PAR. These include Player Efficiency Rating, Win Shares, Box Plus Minus, and 
Value Over Replacement Player
%PAR vs. PER
%PAR vs. WS
%PAR vs. BPM
%PAR vs. VORP

\section{Discussion}
% Analyzing entire study, show how you can use PAR now, did it achieve the overall goal I was looking for?
% Talk about drawbacks, what could be improved, etc.
% Talk about how it is in agreement with existing measures
Ultimately, the goal of this statistic is to allow a reader to gauge a player's value to his team based on a 
league average player. As previously stated, the metric is adjusted so that a league average player has a 
PAR value of 0. Based on all other players' PAR values, we can create a sort of "legend", a rough guideline 
for what a player's role will be based on the threshold his PAR value falls under.

\bibliographystyle{chicago}
\bibliography{citations.bib}
\section*{Acknowledgements}
The authors thank the University of Connecticut and the UConn Department of Statistics.



%Note correct LaTeX quotations above. Do not use the " symbol, but rather double ` followed by double '



% The About the Student Author section is NOT optional.  Write a paragraph about the student; see previous journal editions for examples.
% If there is more than one student author, you must move the comment below.
\section*{About the Student Author}
Brian Krikorian plans to graduate in May 2022 from the University of Connecticut, with a Bachelor's of Science in Data Science, and a minor in Computer Science.

% The Press Summary section is NOT optional.  Write a paragraph describing the paper in a manner suitable for the press; see previous journal editions for examples.
\section*{Press Summary}

\end{document}